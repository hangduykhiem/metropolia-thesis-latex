\chapter{Theory}

Introduction to the theory section

\section{TacoTaco framework}

\subsection{About TacoTaco}

- Developing an android application can be difficult without proper tool set
    - Increasingly big, unorganized codebase
        + Complicated business logic
        + Complicated rendering logic
    - Boilerplate to handle trivial work, such as saving business states, passing variable between activity, etc.
    - Testing for the application would be hard.
    => Toolings and frameworks are needed to maintain a sane project\

- A framework provide you what needed to build a maintainable and extendable project.
    - Reason why A project need to be maintainable and extenable
        - Maintainable => Save later headache
        - Extendable => More features, more funding?
    - Framework provide a set way of doing thing:
        - Add a new component
        - Modify pre-existing component
        - Stucture the dependencies between components, how  components work together
    - Using framework saves a lot of times, headache, makes more testable code
    
- Taco Taco is a modern framework suitable for android development:
    - Get lesson from other existing architecture framework: MVP, MVI, MVVM, Conductor, Google Architecture
    - Is actually used in production in Wolt
    - With these characteristic:
        - Single Activity ( UI Tree based )
        - MVI based
        - Data driven rendering

\subsection{Introduction to TacoTaco}

- TacoTaco is singleActivity:
    - Base on Conductor
    - because navigating with Activities are hard to manages
        - passing states between parcelable might be
        - creating activity is impacting performance
        - Same activity context can be used every
    - Includes multiple controller, interactor and renderer (base component of TacoTaco)
        
- TacoTaco is state and action based:
    - MVI architecutre, with action is the intent (To separate from android intent)
    - every time state update, renderer states update the controller, which sole responsibily is to display the data.
    
- TacoTaco manage your states.
    - main concern of a lot of developer is to manage onSavedInstanceState.
        - List things could go wrong here.
        - Why parcelable is bad.
    - TacoTaco helps solve this.

- TacoTaco is data driven rendering:\
    - Each interactors subscribe to multiple data source
    - states is update when datasource update.
    
- TacoTAco is MVI.

- TacoTaco works well with Dagger. (Hook)

[Insert TacoTaco architecutral picture here]

\subsection{How Taco Taco Works}

- Describe Controller
- Describe Root Controller
- Describe Interactor
- Describe Renderer.
- Describe Analytics
- Describe Transition, Command
- Describe relation with Dagger (What's wrong here: Kapt is very hard to setup)

\subsection{Comparison with other architecture}

- Do further reading here: 
    - https://link.springer.com/book/10.1007%2F978-1-4842-3180-7
    - https://www.researchgate.net/profile/Stephen_Pope/publication/248825145_A_cookbook_for_using_the_model_-_view_controller_user_interface_paradigm_in_Smalltalk_-_80/links/5436c5f30cf2643ab9888926/A-cookbook-for-using-the-model-view-controller-user-interface-paradigm-in-Smalltalk-80.pdf
    - http://www.wildcrest.com/Potel/Portfolio/mvp.pdf
    - https://ieeexplore.ieee.org/abstract/document/469759 (weird model)
    - http://megabyte.utm.ro/articole/2010/info/sem1/InfoStraini_Pdf/1.pdf
    
\section{OCR technology}

- History of OCR:
    - The history of OCR: optical character recognition (1982)
    - http://www.historyofinformation.com/expanded.php?id=885
    - OCR and computer vision


- Using deep learning in OCR

\subsection{OCR model training}

Insert discussion about OCR training, machine learning recognition of character

\subsection{Tessaract and Java API}

Insert technical discussion on the open source Tessaract implmentation, as well as the Java binding.

\clearpage %force the next chapter to start on a new page. Keep that as the last line of your chapter!
