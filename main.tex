%----------------------------------------------------------------------------------------
%	Metropolia Thesis LaTeX Template
%----------------------------------------------------------------------------------------
% License:
% This work is licensed under the Creative Commons Attribution 4.0 International License. 
% To view a copy of this license, visit http://creativecommons.org/licenses/by/4.0/.
%
% However, this license apply to this template. As a template, it is supposed to be 
% modified for your own needs (with your thesis content). For this reason, if you use 
% this project as a template and not specifically distribute it as part of a another 
% package/program, we grant the extra permission to freely copy and modify these files as 
% you see fit and even to delete this copyright notice. 
% In short, you are free to publish your thesis under whatever license you wish, even 
% keep the all rights reserved to you.
%
% Authors:
% Panu Leppäniemi, Patrik Luoto and Patrick Ausderau
%
% Credits:
% Panu Leppäniemi: abstract, def, cleaning,...
% Patrik Luoto: title page, abstract in Finnish, abbreviation, math,...
% Patrick Ausderau: initial version, style, table of content, bibliography, figure, 
%                   appendix, table, source code listing...
%
% Please:
% If you find mistakes, improve this template and alike, please contribute by sharing 
% your improvements and/or send us your feedback there: 
% https://github.com/panunu/metropolia-thesis-latex
% And of course, if you improve it, add yourself as an author.
%
% Compiler:
% Use XeLaTeX as a compiler.
 
%----------------------------------------------------------------------------------------
%	THESIS INFO
%----------------------------------------------------------------------------------------

% All general information (main language, title, author (you), degree programme, major 
% option, etc.)
% Edit the file chapters/0info.tex to change these information
% Global information (title of your thesis, your name, degree programme, major, etc.) 

<<<<<<< HEAD
\def\thesislang{english} %change this depending on the main language of the thesis. 
% "english" is the only other supported language currently. If someone has the swedish, please contribute!

\author{Hang Duy Khiem} %your first name and last name
\def\thesis{Thesis}%keep the half based on the main thesis language
%was Opinnäytetyö

%English section, for title/abstract
\title{Android Optical Character Recognition Solution Based on TacoTaco Framework}
\def\metropoliadegree {Bachelor of Engineering} % change to your needs, e.g. "master", etc.
\def\metropoliadegreeprogramme {Information Technology}
\def\metropoliaspecialisation {Software Engineering}
\def\metropoliainstructors {
Patrick Ausderau, Principal Lecturer
}
\def\metropoliakeywords {android, ocr, tacotaco, framework, }
\date{\longmonth\today}




%----------------------------------------------------------------------------------------
%	GLOBAL STYLES
%----------------------------------------------------------------------------------------

% If you need extra package, etc. modify the style/style.tex file.
% If you are using Windows OS, you will need to change default font to Arial in that 
% style/style.tex file (or install Liberation Sans font to your system).
% If you are using MacOS or linux, make sure you have Liberation Sans font installed.
% Global style. Normally should not be edited. 
% If you use windows OS, eventually change \setmainfont to Arial
% Check around commit https://github.com/panunu/metropolia-thesis-latex/commit/a0c15ac77bab1a52c59c517a18080938e57bf5ef
% to see how the font files were manually added (after downloading them: https://pagure.io/liberation-fonts/ )

\documentclass[11pt,a4paper,oneside,article]{memoir}
\usepackage[\secondlang,\thesislang]{babel}% finnish english swedish
\usepackage{iflang}
\usepackage{amsmath}
\usepackage{amsfonts}
\usepackage{amssymb}
\usepackage{fontspec}
\usepackage{tocloft}
\usepackage{titlesec}
\usepackage[hyphens]{url}
\usepackage{mathtools}
\usepackage{wallpaper}
\usepackage{datetime}
\usepackage[bookmarksdepth=subsection]{hyperref} % for automagic pdf links for toc, refs, etc.
\usepackage[amssymb]{SIunits}
\usepackage[version=3]{mhchem}
\usepackage{pgfplots} %simple plots etc
\usepackage{pgfplotstable}
\usepackage{tikz} % mindmaps, flowcharts, piecharts, examples at http://www.texample.net/tikz/examples/
\usetikzlibrary{shapes.geometric, arrows}


\renewcommand{\dateseparator}{.}
%condition for adding or not space in TOC
\usepackage{etoolbox}
%for compact list
\usepackage{enumitem}
%for block comment
\usepackage{verbatim}
%for "easier" references
\usepackage{varioref}
%forcing single line spacing in bibliography
\DisemulatePackage{setspace}
\usepackage{setspace}
%including figure (image)
\usepackage{graphicx}
%change the numbering for figure
\usepackage{chngcntr}
%strike trough
\usepackage{ulem}
%euro symbol
\usepackage{eurosym}
%try to count
\usepackage{totcount}
%insert source code
%\usepackage{listings}
%require -8bit -shell-escape in the xelatex compile command
%if compiling locally, consider options cachedir=minted,outputdir=~/.tex
\usepackage[newfloat]{minted}
\setminted{tabsize=2,linenos,breaklines,breaksymbolleft={\quad},baselinestretch=1}
\setmintedinline{breaklines}
\usepackage[justification=justified,singlelinecheck=false,font=small]{caption}
\usepackage{color}
%force the width of a table instead of column
\usepackage{tabularx}
\usepackage{booktabs} %why not booktabs? :3
% Abbreviations, acronym and glossary
\usepackage[acronym,toc,nonumberlist,section=chapter]{glossaries}%xindy,%toc, ,nomain
\renewcommand*{\glsclearpage}{}

\usepackage{float} % For forced figure location with modifier H (\begin{figure}[H])
\usepackage{cite} % Make citations to match Metropolia thesis guide

% change font of links in bibliography to same as other text
\usepackage{url}
\urlstyle{same}

% change punctuation of multiple cites to semicolon instead of comma: [1; 2; 3]
\renewcommand\citepunct{; }

% citep-macro for reference with period inside square brackets [1.]
\newcommand{\citep}[1]{
 \renewcommand\citeright{.]}
 \cite{#1}
 \renewcommand\citeright{]}
}

%set date format to D.M.YYYY
\newdateformat{specialdate}{\THEDAY.\THEMONTH.\THEYEAR}
%set date format to D Month YYYY
\newdateformat{longmonth}{\THEDAY~\monthname[\THEMONTH] \THEYEAR}

\newcommand\tn[1]{\textnormal{#1}} %use \tn instead of \textnormal
\newcommand\reaction[1]{\begin{equation}\ce{#1}\end{equation}} %\reaction{} for chemical reactions

%NORMAL TEXT
%all text, title, etc. in the same font: Arial
%NOTE: fontname is case-sensitive
\setmainfont{Liberation Sans}
%line space
\linespread{1.5}
\AtBeginEnvironment{tabular}{\singlespacing}
%\doublespacing
%margin
\usepackage[top=2.5cm, bottom=3cm, left=4cm, right=2cm, nofoot]{geometry}
\setlength{\parindent}{0pt} %first line of paragraph not indented
\setlength{\parskip}{16.5pt} %one empty line to separate paragraph
%list with small line space separation
\tightlists

%IMAGE - FIGURE
%the figures should be placed in the "illustration" folder
\graphicspath{{illustration/}}
%figure number without chapter (1.1, 1.2, 2.1) to (1, 2, 3)
\counterwithout{figure}{chapter}
%border around images
\setlength\fboxsep{0pt}
\setlength\fboxrule{0.5pt}
%space after figure caption (and other float elements)
\setlength{\belowcaptionskip}{-7pt}

%TABLE
\counterwithout{table}{chapter}

%SOURCE CODE
\newenvironment{code}{\captionsetup{type=listing}}{}
\IfLanguageName {finnish} {\SetupFloatingEnvironment{listing}{name=Koodiesimerkki}} {}%was Listaus
%\counterwithout{lstlisting}{chapter}
%moved after begin document, otherwise does not compile

%% set this format as the default for lstlisting
%\DeclareCaptionFormat{empty}{}
%\captionsetup[lstlisting]{format=empty}

%TOC
%change toc title
\IfLanguageName {finnish} {\addto{\captionsfinnish}{\renewcommand*{\contentsname}{Sisällys}}} {}
%remove dots
\renewcommand*{\cftdotsep}{\cftnodots}
%chapter title and page number not in bold
\renewcommand{\cftchapterfont}{}
\renewcommand{\cftchapterpagefont}{}
%sub section in toc
\setcounter{tocdepth}{2}
%subsection numbered
\setcounter{secnumdepth}{2}
\renewcommand{\tocheadstart}{\vspace*{-15pt}}
\renewcommand{\printtoctitle}[1]{\fontsize{13pt}{13pt}\bfseries #1}
%\renewcommand{\aftertoctitle}{\vspace*{-22pt}\afterchaptertitle}
%spacing afer a chapter in toc
\preto\section{%
  \ifnum\value{section}=0\addtocontents{toc}{\vskip11pt}\fi
}
%spacing afer a section in toc
\renewcommand{\cftsectionaftersnumb}{\vspace*{-3pt}}
%spacing afer a subsection in toc
\renewcommand{\cftsubsectionaftersnumb}{\vspace*{-1pt}}
%appendix in toc with "Appendix " + num
\IfLanguageName {finnish} {
  \renewcommand*{\cftappendixname}{Liite\space}
  \renewcommand{\appendixtocname}{Liitteet}
}{\renewcommand*{\cftappendixname}{Appendix\space}}
%appendix header
\IfLanguageName {finnish} {\def\appname{Liite\space}}{\def\appname{Appendix\space}}

%TITLES
%chapter title
%\clearforchapter{\clearpage}
\titleformat{\chapter}
{\fontsize{13pt}{13pt}\bfseries\linespread{1}}%\clearpage
{\thechapter}{.5cm}{}
\titlespacing*{\chapter}{0pt}{.32cm}{9pt}
\titleformat{\section}
{\fontsize{12pt}{12pt}\linespread{1}}
{\thesection}{.5cm}{}
\titlespacing*{\section}{0pt}{14pt}{6pt}
\titleformat{\subsection}
{\fontsize{12pt}{12pt}\linespread{1}}
{\thesubsection}{.5cm}{}
\titlespacing*{\subsection}{0pt}{14pt}{6pt}


%QUOTE
\renewenvironment{quote}
  {\list{}{\rightmargin=0pt\leftmargin=1cm\topsep=-10pt}%
  \item\relax\fontsize{10pt}{10pt}\singlespacing}
  {\endlist}

%BIBLIOGRAPHY
%bibliography title to be "references"
%IF THE TITLE DON'T GET RENAMED PROPERLY, move that line after the \begin{document}
\IfLanguageName {finnish} {\addto{\captionsfinnish}{\renewcommand*{\bibname}{Lähteet}}} {\renewcommand\bibname{References}}
\makeatletter %reference list option change
\renewcommand\@biblabel[1]{#1\hspace{1cm}} %from [1] to 1 with 1cm gap
\makeatother %
\setlength{\bibitemsep}{11pt}

%count the appendices (since the chapter counter is reset after \appendix).
%! require to complie 2 times
\regtotcounter{chapter}


\makepagestyle{tiivis}
\makeevenhead{tiivis}{}{}{Tiivistelmä}
\makeoddhead{tiivis}{}{}{Tiivistelmä}

\makepagestyle{abstract}
\makeevenhead{abstract}{}{}{Abstract}
\makeoddhead{abstract}{}{}{Abstract}

%footer on every pages
\LLCornerWallPaper{1}{\IfLanguageName{finnish}{footer_fi}{footer_en}}

% Normally, you do not need to modify the title style. It's content comes from the 
% chapters/0info.tex file.
% TITLE PAGE
% Normally, you should not edit this file.

\makeatletter
\renewcommand{\maketitle}{
\newgeometry{left=4.5cm}
\thispagestyle{empty}
\ThisULCornerWallPaper{1}{\IfLanguageName{finnish}{metropolia_fi}{metropolia_en}}
%
\vspace*{12.8cm}
\tn{\Large\@author\\[1cm]\Huge\IfLanguageName {finnish}{\color[RGB]{155,50,35}\otsikko}{\color[RGB]{31,73,125}\@title}}%\\[22pt]\LARGE\alaotsikko\\[1.75cm]}

\null\vfill
\parbox{.7\linewidth}{
\IfLanguageName {finnish}{
  Metropolia Ammattikorkeakoulu\\
  \tutkinto \\
  \kohjelma \\
  \thesisfi\\
  \pvm
} {
  Helsinki Metropolia University of Applied Sciences\\
  \metropoliadegree \\
  \metropoliadegreeprogramme \\
  \thesisen\\
  \IfLanguageName {finnish}{\pvm}{\@date} % D.M.YYYY date format for Finnish. D Month YYYY for English
}
}
\vspace*{-1.5cm}
\clearpage
\restoregeometry
}
\makeatother



%----------------------------------------------------------------------------------------
%	ABBREVIATION AND GLOSSARY
%----------------------------------------------------------------------------------------

% Add/edit all your acronyms, abbreviations, glossary entries, etc. definitions in 
% chapters/0abbr.tex file.
% You can have as many as you wish. Only the ones you use in your text (inserted with 
% \gls{} command) will print in the Glossary/Lyhenteet.
\input{chapters/0abbr.tex}

%----------------------------------------------------------------------------------------
%	DOCUMENT STARTS HERE...
%----------------------------------------------------------------------------------------

\begin{document}
\counterwithout{listing}{chapter}

%----------------------------------------------------------------------------------------
%	TITLE PAGE
%----------------------------------------------------------------------------------------

\input{style/title_headers.tex}
\maketitle
\newpage
%all abstract, table of content and glossary will get the metropolia logo at bottom
\LRCornerWallPaper{1}{footer}

%----------------------------------------------------------------------------------------
%	ABSTRACT / Tiivistelmä
%----------------------------------------------------------------------------------------

% If you are international student writing in English, remove the Finnish abstract.
% If you are Finnish citizen, you must have 2 abstracts, one in Finnish (or Swedish 
% depending on your mother tongue) and one in English regardless of the main language of 
% your thesis.
% Abstract in English
%Most probably, you only need to change the text of the abstract. Everything else comes from chapter/0info.tex
%If you do not have any appendix, you may delete \total{chapter} and replace with 0

\pagestyle{abstract}
\begin{otherlanguage}{english}
{\renewcommand{\arraystretch}{2}%
\begin{tabular}{ | p{4,7cm} | p{10,3cm} |}
  \hline
  Author(s) \newline
  Title \newline\newline 
  Number of Pages \newline
  Date
  & 
  \makeatletter
  \@author \newline
  \@title \newline\newline %! if very long title over 2 lines, remove one \newline
  \pageref*{LastPage} pages + \total{chapter} appendices \newline %! if no appendices, risk to count total of chapter :D
  \IfLanguageName {finnish} {\foreignlanguage{english}{\longdate\@date}} {\@date}
  \makeatother
  \\ \hline
  Degree & \metropoliadegree
  \\ \hline
  Degree Programme & \metropoliadegreeprogramme
  \\ \hline
  Professional Major & \metropoliaspecialisation
  \\ \hline
  Instructor(s) & \metropoliainstructors
  \\ \hline
  \multicolumn{2}{|p{15cm}|}{\vspace{-22pt}
  TODO: \newline

  Insert content here when thesis is ready.
  } \\[14cm] \hline
  Keywords & \metropoliakeywords
  \\ \hline
\end{tabular}
}
\end{otherlanguage}
\clearpage



%----------------------------------------------------------------------------------------
%	License? Acknowledgement?
%----------------------------------------------------------------------------------------

% Uncomment next line and edit chapters/0license.tex if you want license in your thesis.
%\input{chapters/0license.tex}

% Uncomment next line and edit chapters/0acknowledgement.tex if you want acknowledgements.
%\input{chapters/0acknowledgement.tex}

%----------------------------------------------------------------------------------------
%	TABLE OF CONTENTS
%----------------------------------------------------------------------------------------

\input{style/toc.tex}

%list of figure, tables would come here if relevant?

%----------------------------------------------------------------------------------------
%	Lyhenteet / Abbreviation
%----------------------------------------------------------------------------------------

% If you don't use abbreviations/glossary, remove the following line.
% Abbreviation and Glossary
% Normally, you don't have to modify this file. Your abbreviations, etc. goes in 
% ../chapters/0abbr.tex file.

\begin{singlespacing}

% \gsladdall would add all terms even if not used in your text.
%\glsaddall
  \addtocontents{toc}{\cftpagenumbersoff{chapter}}
{
	\titleformat{\section}
	{\fontsize{13pt}{13pt}\bfseries\linespread{1}}
	{\thesection}{.5cm}{}
	%Adapt labelwidth (sorry for the ugly hack)
	\setlist[description]{leftmargin=!, labelwidth=4em}
	\IfLanguageName {finnish} {
		\printacronyms[title=Lyhenteet]
	}{
		\printacronyms[title=List of Abbreviations]
	}
	\setlist[description]{leftmargin=!, labelwidth=7em}
	\printglossary 
	\setlist[description]{style=standard} % reset settings back to default
}
\addtocontents{toc}{\cftpagenumberson{chapter}} 
\end{singlespacing}

\clearpage


%----------------------------------------------------------------------------------------
%	CONTENT
%----------------------------------------------------------------------------------------

% Style for the main thesis content.
% Normally, you should not edit this file.

%page number always on top right; also for chapter "title" page
\pagestyle{plain}
\makeevenhead{plain}{}{}{\thepage}
\makeoddhead{plain}{}{}{\thepage}

\setcounter{page}{1} %page 1 should be Introduction

\sloppy % enforce alignment to fully justified


%reset page number to 1, no more logo footer, etc.

% Introduction

\chapter{Introduction}

- Introduction to the topic of languages. Link this to the current technology and OCR. 



- On another topic, everybody has a phone => could very well be use as a language center of humanity.

- This project aim to demonstrate the capacity of OCR using the framework + the tech.

TODO: Insert Introduction here

\clearpage
% Project Specifications

\chapter{Project Specifications}

- Goal of the project: Even though there already exist solutions to solve the problem of OCR on Android, the project helps identify, analyze and study the concept of building an android application using TacoTaco and implementing OCR technology, through a practical steps by steps project.

- Here: listing the current state of Android OCR:
    - Google Translate
    - Android OCR
    
- Here: listing the current state of Android framework:
    - MVI
    - MVVM
    - MVP

The project will include:
- Project plan
- OCR training model
- Android application that reads and print text.
- Documentation of project building

The project will not include:
- App publisihing?
- 

\clearpage %force the next chapter to start on a new page. Keep that as the last line of your chapter!

\chapter{Theory}

Introduction to the theory section

\section{TacoTaco framework}

\subsection{About TacoTaco}

- Developing an android application can be difficult without proper tool set
    - Increasingly big, unorganized codebase
        + Complicated business logic
        + Complicated rendering logic
    - Boilerplate to handle trivial work, such as saving business states, passing variable between activity, etc.
    - Testing for the application would be hard.
    => Toolings and frameworks are needed to maintain a sane project\

- A framework provide you what needed to build a maintainable and extendable project.
    - Reason why A project need to be maintainable and extenable
        - Maintainable => Save later headache
        - Extendable => More features, more funding?
    - Framework provide a set way of doing thing:
        - Add a new component
        - Modify pre-existing component
        - Stucture the dependencies between components, how  components work together
    - Using framework saves a lot of times, headache, makes more testable code
    
- Taco Taco is a modern framework suitable for android development:
    - Get lesson from other existing architecture framework: MVP, MVI, MVVM, Conductor, Google Architecture
    - Is actually used in production in Wolt
    - With these characteristic:
        - Single Activity ( UI Tree based )
        - MVI based
        - Data driven rendering

\subsection{Introduction to TacoTaco}

- TacoTaco is singleActivity:
    - Base on Conductor
    - because navigating with Activities are hard to manages
        - passing states between parcelable might be
        - creating activity is impacting performance
        - Same activity context can be used every
    - Includes multiple controller, interactor and renderer (base component of TacoTaco)
        
- TacoTaco is state and action based:
    - MVI architecutre, with action is the intent (To separate from android intent)
    - every time state update, renderer states update the controller, which sole responsibily is to display the data.
    
- TacoTaco manage your states.
    - main concern of a lot of developer is to manage onSavedInstanceState.
        - List things could go wrong here.
        - Why parcelable is bad.
    - TacoTaco helps solve this.

- TacoTaco is data driven rendering:\
    - Each interactors subscribe to multiple data source
    - states is update when datasource update.
    
- TacoTAco is MVI.

- TacoTaco works well with Dagger. (Hook)

[Insert TacoTaco architecutral picture here]

\subsection{How Taco Taco Works}

- Describe Controller
- Describe Root Controller
- Describe Interactor
- Describe Renderer.
- Describe Analytics
- Describe Transition, Command
- Describe relation with Dagger (What's wrong here: Kapt is very hard to setup)

\subsection{Comparison with other architecture}

- Do further reading here: 
    - https://link.springer.com/book/10.1007%2F978-1-4842-3180-7
    - https://www.researchgate.net/profile/Stephen_Pope/publication/248825145_A_cookbook_for_using_the_model_-_view_controller_user_interface_paradigm_in_Smalltalk_-_80/links/5436c5f30cf2643ab9888926/A-cookbook-for-using-the-model-view-controller-user-interface-paradigm-in-Smalltalk-80.pdf
    - http://www.wildcrest.com/Potel/Portfolio/mvp.pdf
    - https://ieeexplore.ieee.org/abstract/document/469759 (weird model)
    - http://megabyte.utm.ro/articole/2010/info/sem1/InfoStraini_Pdf/1.pdf
    
\section{OCR technology}

- History of OCR:
    - The history of OCR: optical character recognition (1982)
    - http://www.historyofinformation.com/expanded.php?id=885
    - OCR and computer vision


- Using deep learning in OCR

\subsection{OCR model training}

Insert discussion about OCR training, machine learning recognition of character

\subsection{Tessaract and Java API}

Insert technical discussion on the open source Tessaract implmentation, as well as the Java binding.

\clearpage %force the next chapter to start on a new page. Keep that as the last line of your chapter!

% Proposed solution

\chapter{Proposed solution}

Implementation of the project here

\clearpage %force the next chapter to start on a new page. Keep that as the last line of your chapter!

% Conclusions

\chapter{Discussion}

Insert Discussion here

\clearpage %force the next chapter to start on a new page. Keep that as the last line of your chapter!

\input{chapters/conclusion.tex}


%----------------------------------------------------------------------------------------
%	BIBLIOGRAPHY REFERENCES
%----------------------------------------------------------------------------------------

\input{style/biblio.tex}

%----------------------------------------------------------------------------------------
%	APPENDICES 
%----------------------------------------------------------------------------------------

\input{style/appendix.tex}
%force smaller vertical spacing in table of content
%!!! There can be some fun depending if the appendices have (sub)sections or not :D
% You will have to play with these numbers and eventually add the \vspace line  before 
% some \chapter and force another number.
% To add more fun, time to time the table of content get wrong after a build :(

\liite{1}% This is a hack to have right page numbering for each appendix. Make sure to 
	 % use a unique number for each appendix.
% are safe to delete this lines (and the next samples) once you refreshed your LaTeX 
% skills (and safe to delete the sample folder and all its file too).

%----------------------------------------------------------------------------------------
%	THIS IS THE END 
%----------------------------------------------------------------------------------------
\end{document}
